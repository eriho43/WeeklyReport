% !TeX spellcheck = en_US
%\documentstyle[epsf,twocolumn]{jarticle}       %LaTeX2e仕様

%\documentclass[twocolumn]{jarticle}     %pLaTeX2e仕様(platex.exeの場合)

%\documentclass[twocolumn]{ujarticle}     %pLaTeX2e仕様(uplatex.exeの場合)

%\documentclass[11pt,a4paper,uplatex]{ujarticle} 	% for uplatex

\documentclass[11pt,a4j]{jarticle} 			% for platex


%%%%%%%%%%%%%%%%%%%%%%%%%%%%%%%%%%%%%%%%%%%%%%%%%%%%%%%%%%%%%%

%%

%%  基本バージョン

%%

%%%%%%%%%%%%%%%%%%%%%%%%%%%%%%%%%%%%%%%%%%%%%%%%%%%%%%%%%%%%%%%%

\setlength{\topmargin}{-45pt}

%\setlength{\oddsidemargin}{0cm} 

\setlength{\oddsidemargin}{-7.5mm}

%\setlength{\evensidemargin}{0cm} 

\setlength{\textheight}{24.1cm}

%setlength{\textheight}{25cm} 

\setlength{\textwidth}{17.4cm}

%\setlength{\textwidth}{172mm} 

\setlength{\columnsep}{11mm}

%\usepackage[dvipdfmx]{graphicx}


%\kanjiskip=.07zw plus.5pt minus.5pt



% 【節が変わるごとに (1.1)(1.2) … (2.1)(2.2) と数式番号をつけるとき】

%\makeatletter

%\renewcommand{\theequation}{%

%\thesection.\arabic{equation}} %\@addtoreset{equation}{section}

%\makeatother


%\renewcommand{\arraystretch}{0.95} 行間の設定


%%%%%%%%%%%%%%%%%%%%%%%%%%%%%%%%%%%%%%%%%%%%%%%%%%%%%%%%



\usepackage[dvipdfmx]{graphicx} 
 %pLaTeX2e仕様(\documentstyle ->\documentclass)
\usepackage{listings}
 
 
 \lstset{
 	basicstyle={\ttfamily},
 	identifierstyle={\small},
 	commentstyle={\smallitshape},
 	keywordstyle={\small\bfseries},
 	ndkeywordstyle={\small},
 	stringstyle={\small\ttfamily},
 	frame={tb},
 	breaklines=true,
 	columns=[l]{fullflexible},
 	numbers=left,
 	xrightmargin=0zw,
 	xleftmargin=3zw,
 	numberstyle={\scriptsize},
 	stepnumber=1,
 	numbersep=1zw,
 	lineskip=-0.5ex
 }

%%%%%%%%%%%%%%%%%%%%%%%%%%%%%%%%%%%%%%%%%%%%%%%%%%%%%%%%


\begin{document}

	
	%\twocolumn[

	\noindent

	
	\hspace{1em}

	2021年11月19日(金)ゼミ資料

	\hfill
	
	\vspace{2mm}

	
	\hrule

	
	\begin{center}

		{\Large \bf 進捗報告}

	\end{center}

	\hrule

	\vspace{3mm}

	%]
	
	\section{superpixelを用いた多義箇所探索コードを書くに当たって詰まっているところ}
	以下コードのslicの戻り値(コード中ではslic\_segments)の示す対象がわからない.公式ドキュメントより,
	Returns:
	labels2D or 3D array.
	Integer mask indicating segment labels.
	とあるが,何のことなのか理解できていない.
	
	中身のデータは,
	[[0 0 0 0 0 0 0 0 0 0 0 0 0 0 0 0 0 0 0 0 0 0 0 0 0 0 0 0 0 0 0 0 0 0 0 0 0
	0 0 0 0 0 0 0 0 0 0 0 0 0 0 0 0 0 0 0 0 0 0 0 0 0 0 0 0 0 0 0 0 0 0 0 0 0
	0 0 0 0 0 0 0 0 0 0 0 0 1 1 1 1 1 1 1 1 1 1 1 1 1 1 1 1 1 1 1 1 1 1 1 1 1
	1 1 1 1 1 1 1 1 1 1 1 1 1 1 1 1 1 1 1 1 1 1 1 1 1 1 1 1 1 1 1 1 1 1 1 1 1
	1 1 1 1 2 2 2 2 2 2 2 2 2 2 2 2 2 2 2 2 2 2 2 2 2 2 2 2 2 2 2 2 2 2 2 2 2
	2 2 2 2 2 2 2 2 2 2 2 2 2 2 2 2 2 2 2 2 2 2 2 2 2 2 2 2 2 2 2 2 2 2 2 2 2
	2 2 2 2 2 2 2 2 2 2 2 2 2 2 2 2 2 2 2 2 2 2 2 2 2 2 2 2 2 2 2 3 3 3 3 3 4
	4 4 4 4 4 4 4 4 4 4 4 4 4 4 4 4 4 4 4 4 4 4 4 4 4 4 4 4 4 4 4 4 4 4 4 4 4
	4 4 4 4 4 4 4 4 4 4 4 4 4 4 4 4 4 4 4 4 4 4 4 4 4 4 4 4 4 4 4 4 4 4 4 4 4
	4 4 4 4 4 4 4 4 4 4 4 4 4 4 4 4 4 4 4 4 4 4 4 4 4 4 4 4 4 4 4 4 4 4 4 4 4
	4 4 4 4 4 4 4 4 4 4 4 4 4 4 4 4 4 4 4 4 4 4 4 4 4 4 4 4 4 4 4 4 4 4 4 4 5
	5 5 5 5 5 5 5 5 5 5 5 5 5 5 5 5 5 5 5 5 5 5 5 5 5 5 5 5 5 5 5 5 5 5 5 5 5
	5 5 5 5 5 5 5 5 5 5 5 5 5 5 5 5 5 5 5 5 5 5 5 5 5 5 5 5 5 5 5 5 5 5 5 5 5
	5 5 5 5 5 5 5 5 5 5 5 5 5 5 5 5 5 5 5 5 5 5 5]
	[0 0 0 0 0 0 0 0 0 0 0 0 0 0 0 0 0 0 0 0 0 0 0 0 0 0 0 0 0 0 0 0 0 0 0 0 0
	0 0 0 0 0 0 0 0 0 0 0 0 0 0 0 0 0 0 0 0 0 0 0 0 0 0 0 0 0 0 0 0 0 0 0 0 0
	0 0 0 0 0 0 0 0 0 0 0 1 1 1 1 1 1 1 1 1 1 1 1 1 1 1 1 1 1 1 1 1 1 1 1 1 1
	1 1 1 1 1 1 1 1 1 1 1 1 1 1 1 1 1 1 1 1 1 1 1 1 1 1 1 1 1 1 1 1 1 1 1 1 1
	1 1 1 1 2 2 2 2 2 2 2 2 2 2 2 2 2 2 2 2 2 2 2 2 2 2 2 2 2 2 2 2 2 2 2 2 2
	2 2 2 2 2 2 2 2 2 2 2 2 2 2 2 2 2 2 2 2 2 2 2 2 2 2 2 2 2 2 2 2 2 2 2 2 2
	2 2 2 2 2 2 2 2 2 2 2 2 2 2 2 2 2 2 2 2 2 2 2 2 2 2 2 2 2 2 2 3 3 3 3 3 4
	4 4 4 4 4 4 4 4 4 4 4 4 4 4 4 4 4 4 4 4 4 4 4 4 4 4 4 4 4 4 4 4 4 4 4 4 4
	4 4 4 4 4 4 4 4 4 4 4 4 4 4 4 4 4 4 4 4 4 4 4 4 4 4 4 4 4 4 4 4 4 4 4 4 4
	4 4 4 4 4 4 4 4 4 4 4 4 4 4 4 4 4 4 4 4 4 4 4 4 4 4 4 4 4 4 4 4 4 4 4 4 4
	4 4 4 4 4 4 4 4 4 4 4 4 4 4 4 4 4 4 4 4 4 4 4 4 4 4 4 4 4 4 4 4 4 4 4 4 4
	5 5 5 5 5 5 5 5 5 5 5 5 5 5 5 5 5 5 5 5 5 5 5 5 5 5 5 5 5 5 5 5 5 5 5 5 5
	5 5 5 5 5 5 5 5 5 5 5 5 5 5 5 5 5 5 5 5 5 5 5 5 5 5 5 5 5 5 5 5 5 5 5 5 5
	5 5 5 5 5 5 5 5 5 5 5 5 5 5 5 5 5 5 5 5 5 5 5]
	といった二次元配列が,504*644出力されている.
	また,slic\_segmentsをそのままplotすると,以下のような画像が出力される.
	
	\begin{lstlisting}[caption=slicアルゴリズムのコード例,label=fuga]
	from skimage import io
	from skimage.segmentation import felzenszwalb, quickshift, slic, watershed, mark_boundaries
	import matplotlib.pyplot as plt
	
	img = io.imread("path.jpg")
	
	slic_segments = slic(img, n_segments=100)
	\end{lstlisting}
	

	
\begin{figure}
	\centering
	\includegraphics[width=0.4\linewidth]{2e}
	\caption{元画像}
	\label{fig:2e}
\end{figure}
\begin{figure}
	\centering
	\includegraphics[width=0.7\linewidth]{slic_boumdary}
	\caption{slic結果}
	\label{fig:slicboumdary}
\end{figure}
\begin{figure}
	\centering
	\includegraphics[width=0.7\linewidth]{slic}
	\caption{slic\_segmentsのplot結果}
	\label{fig:slic}
\end{figure}

\section{多義図形データセット増強}
コードを回している時間に多義図形データセットを増強した.顔と風景の多義図形画像が,365枚→390枚となった.

\section{mnistによるautokerasの予備実験}
自前データでautokerasを回した際,1epoch目からval\_loss=0, val\_acc=1となってしまうバグが頻発した.
原因を探るため,まずは36枚しか無かったvalidデータを130枚に増やして再実験したが,同じバグが起きた.
よって,mnistをデータセットとして予備実験を試したが,またしても同じエラーが起きているため,autokerasライブラリ自体(もしくはkeras, tensorflow?)のバグであると考えられる.

\end{document}



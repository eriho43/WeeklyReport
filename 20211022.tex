%\documentstyle[epsf,twocolumn]{jarticle}       %LaTeX2e仕様

%\documentclass[twocolumn]{jarticle}     %pLaTeX2e仕様(platex.exeの場合)

%\documentclass[twocolumn]{ujarticle}     %pLaTeX2e仕様(uplatex.exeの場合)

%\documentclass[11pt,a4paper,uplatex]{ujarticle} 	% for uplatex

\documentclass[11pt,a4j]{ujarticle} 			% for platex


%%%%%%%%%%%%%%%%%%%%%%%%%%%%%%%%%%%%%%%%%%%%%%%%%%%%%%%%%%%%%%

%%

%%  基本バージョン

%%

%%%%%%%%%%%%%%%%%%%%%%%%%%%%%%%%%%%%%%%%%%%%%%%%%%%%%%%%%%%%%%%%

\setlength{\topmargin}{-45pt}

%\setlength{\oddsidemargin}{0cm} 

\setlength{\oddsidemargin}{-7.5mm}

%\setlength{\evensidemargin}{0cm} 

\setlength{\textheight}{24.1cm}

%setlength{\textheight}{25cm} 

\setlength{\textwidth}{17.4cm}

%\setlength{\textwidth}{172mm} 

\setlength{\columnsep}{11mm}

%\usepackage[dvipdfmx]{graphicx}


%\kanjiskip=.07zw plus.5pt minus.5pt



% 【節が変わるごとに (1.1)(1.2) … (2.1)(2.2) と数式番号をつけるとき】

%\makeatletter

%\renewcommand{\theequation}{%

%\thesection.\arabic{equation}} %\@addtoreset{equation}{section}

%\makeatother


%\renewcommand{\arraystretch}{0.95} 行間の設定


%%%%%%%%%%%%%%%%%%%%%%%%%%%%%%%%%%%%%%%%%%%%%%%%%%%%%%%%

\usepackage[dvipdfmx]{graphicx}  %pLaTeX2e仕様(\documentstyle ->\documentclass)

%%%%%%%%%%%%%%%%%%%%%%%%%%%%%%%%%%%%%%%%%%%%%%%%%%%%%%%%


\begin{document}

	
	%\twocolumn[

	\noindent

	
	\hspace{1em}

	2021年10月22日(金)ゼミ資料

	\hfill
	
	\vspace{2mm}

	
	\hrule

	
	\begin{center}

		{\Large \bf 進捗報告}

	\end{center}

	\hrule

	\vspace{3mm}

	%]
	
	\section{autokerasの実験結果}
	
	
	\begin{table}[h]
		\caption{実験条件}
		\label{}
		\centering
		\begin{tabular}{c|c}
			\hline
			クラス&3クラス(多義図形,風景画,肖像画)\\ \hline
			エポック&Early Stopping\\ \hline
			訓練枚数&2570枚/クラス\\ \hline
			評価枚数&36枚/クラス\\ \hline
			データサイズ&200×200×3(RGB)\\ \hline
			損失関数&交差エントロピー\\ \hline
			試行回数&100回\\ \hline
		\end{tabular}
	\end{table}



	\begin{table}[h]
		\caption{モデル構造}
		\label{}
		\centering
		\begin{tabular}{c|c}
			\hline
			image block 1/normalize&True\\ \hline              
			image block 1/augment&False\\ \hline    
			image block 1/block type&vanilla\\ \hline         
			classification head 1/spatial reduction 1/reduction type&flatten\\ \hline
			cclassification head 1/dropout&0.5\\ \hline
			optimizer&adam\\ \hline              
			learning rate&0.001\\ \hline
			image block 1/conv block 1/kernel size&3\\ \hline
			image block 1/conv block 1/separable&False\\ \hline
			image block 1/conv block 1/max pooling&True\\ \hline
			image block 1/conv block 1/dropout&0.25\\ \hline
			image block 1/conv block 1/num blocks&1\\ \hline
			image block 1/conv block 1/num layers&2\\ \hline
			image block 1/conv block 1/filters 0 0&32\\ \hline
			image block 1/conv block 1/filters 0 1&64\\ \hline
		\end{tabular}
	\end{table}
	
	loss: 147.65377807617188
	accuracy: 0.3333333432674408 \\
	lossが下がらずaccuracyが上がってない理由は,探索途中でval lossが何らかのバグで0になっており,それを正として探索を進めているからであると考えられる.
	
	\section{今後の方針}
	LIMEによって判断根拠箇所をマスクした状態での実験,autokeras, LIMEの理論調査,\textbf{PCを直す!!!!!}
\end{document}


